\documentclass[12pt]{ltjsarticle} % 日本語用のクラスファイル
% --- 基本パッケージ ---
\usepackage[utf8]{inputenc}
\usepackage{luatexja}  % 日本語を使うためのパッケージ
\usepackage{amsmath, amssymb} % 数式用
\usepackage{graphicx} % 画像挿入用
\usepackage{geometry}  % 余白設定

% --- 表のデザインを向上させるパッケージ ---
\usepackage{tikz}      % 図形などを描きたい場合
\usepackage{booktabs} % 横螺旋を綺麗にする
\usepackage{tabularx} % 幅を指定できる表を作成
\usepackage{longtable} % 複数ページにまたがる表を作成
\usepackage{array} % 表の列の書式を拡張

% --- セクションタイトルのデザイン調整 ---
\usepackage{titlesec}
\titleformat{\section}{\normalfont\Large\bfseries}{\thesection}{1em}{}
\titleformat{\subsection}{\normalfont\large\bfseries}{\thesubsection}{1em}{}
\titleformat{\subsubsection}{\normalfont\normalsize\bfseries}{\thesubsubsection}{1em}{}

% --- 箇条書きのデザイン調整 ---
\usepackage{enumitem}

% --- ハイパーリンク設定 ---
\usepackage[
colorlinks=true,
linkcolor=blue,
urlcolor=cyan,
citecolor=green
]{hyperref}

% --- その他 ---
\usepackage{listings}  % ソースコードを表示するためのパッケージ
\usepackage{xcolor}  % 文字色指定
\usepackage{float} % preambleに追加
\geometry{top=2cm, bottom=2cm, left=2cm, right=2cm}
\usepackage{minted}

\title{要件定義書}
\author{saku0512}
\date{}

\begin{document}
\maketitle

\setcounter{tocdepth}{3}
\tableofcontents % 目次を生成

\newpage

\section{概要}

本プロジェクトは、学校のスポーツ大会運営を効率化するためのWebアプリケーションを開発するものである。大会の準備から結果の記録までを一元管理する\textbf{管理者用アプリ}と、学生が大会情報を確認し、参加登録を行うための\textbf{学生用アプリ}の2つを開発する。

これにより、実行委員会の負担を軽減し、学生への情報伝達をスムーズにすることを目的とする。

\section{プロジェクト構成}

\begin{itemize}
    \item \textbf{モノリポ(Monorepo)}: 一つのGitリポジトリで管理者用、学生用、バックエンドの3つのプロジェクトを管理する。これにより、コードの共通化やバージョン管理が容易になる。
        \begin{itemize}
            \item \texttt{packages/front/admin-app}: 管理者用フロントエンド (Svelte 5)
            \item \texttt{packages/front/student-app}: 学生用フロントエンド (Svelte 5)
            \item \texttt{packages/back}: 共通バックエンド (Gin + GraphQL + MySQL)
        \end{itemize}
    \item \textbf{アプリケーション}:
        \begin{enumerate}
            \item \textbf{管理者用アプリ}: 大会情報の登録、更新、管理を行うWebアプリ。実行委員のみが利用。
            \item \textbf{学生用アプリ}: 大会情報の閲覧、および参加登録用QRコードの発行を行うWebアプリ。全学生が利用。
        \end{enumerate}
\end{itemize}

\section{技術スタック}

\begin{table}[H]
    \centering
    \caption{技術スタック一覧}
    \label{tab:tech_stack}
    \begin{tabular}{lp{4cm}p{7cm}}
        \toprule
        \textbf{領域} & \textbf{技術・ライブラリ} & \textbf{理由} \\
        \midrule
        バックエンド & Go (Gin) & 高速なパフォーマンスとシンプルな文法が特徴。小〜中規模のAPIサーバー開発に適しており、今回の要件にマッチする。 \\
        \addlinespace
        API & GraphQL (gqlgen) & \textbf{推奨}。トーナメント表や試合結果など、関連するデータを一度に効率よく取得する場面が多いため。フロントエンドが必要なデータだけをリクエストでき、REST APIで起こりがちな過剰なデータ取得(オーバーフェッチ)や複数回のリクエスト(N+1問題)を防げる。 \\
        \addlinespace
        フロントエンド & Svelte 5 & コンパイル時に最適化されるため、高速な動作が期待できる。シンプルで学習コストが低く、迅速な開発に適している。 \\
        \addlinespace
        認証 & Microsoft Authentication / Google Authentication & OAuth 2.0を利用したセキュアな認証を実装する。学校で利用しているGoogle WorkspaceやMicrosoft 365のアカウントを流用することで、ユーザー管理の負担を軽減する。 \\
        \addlinespace
        DB & PostgreSQL / MySQL (選択) & リレーショナルデータとの相性が良く、信頼性が高い。 \\
        \addlinespace
        インフラ & Docker & コンテナ化を利用した独立環境を構築することで、環境の整合性を高め、コードの構築やテストを簡単に行うことが可能になる。 \\
        \bottomrule
    \end{tabular}
\end{table}

\section{ユーザー種別と権限}

\begin{table}[H]
    \centering
    \caption{ユーザー権限}
    \label{tab:permissions}
    \begin{tabular}{lcc}
        \toprule
        \textbf{機能} & \textbf{管理者(実行委員)} & \textbf{学生} \\
        \midrule
        アプリへのログイン & ✓ 両方 & ✓ 学生用のみ \\
        スポーツ情報の登録・編集 & ✓ & X \\
        競技概要の登録・編集 & ✓ & X \\
        参加学生の登録 & ✓ & X \\
        試合結果の入力 & ✓ & X \\
        トーナメント表の閲覧 & ✓ & ✓ 閲覧のみ \\
        学生情報の管理 & ✓ & X \\
        参加登録用QRコード読取 & ✓ & X \\
        参加登録用QRコード発行 & X (※1) & ✓ \\
        \bottomrule
    \end{tabular}
    \begin{flushleft}
        (※1) 管理者も学生として競技に参加する場合を考慮し、学生用アプリにログインすれば発行可能。
    \end{flushleft}
\end{table}


\section{共通機能要件}

\subsection{認証機能}
\begin{itemize}
    \item GoogleまたはMicrosoftアカウントによるOAuth認証を実装する。
    \item \textbf{ドメイン制限}: 学校から指定されたドメイン( \texttt{@sendai-nct.jp})を持つアカウントのみログインを許可する。
    \item \textbf{ホワイトリスト}: 上記ドメイン制限に加え、事前に登録されたメールアドレスリスト(ホワイトリスト)に含まれるユーザーのみがログインできるようにする。これにより、関係者以外のアクセスを完全に遮断する。
\end{itemize}

\subsection{初回ログイン時の表示名設定}
\begin{itemize}
    \item いずれのアプリでも、ユーザーが初めてログインした際に、アプリ内で使用する\textbf{表示名}を設定する画面にリダイレクトする。
    \item ダッシュボードなどのメイン機能は、表示名が設定されるまで利用できないようにする。
    \item 設定した表示名は、後からプロフィールページ(要追加検討)で変更できるようにする。
\end{itemize}

\section{管理者用アプリ機能要件}

\subsection{ダッシュボード}
登録済みのスポーツ一覧、本日の試合予定、未入力の試合結果などを一覧表示する。

\subsection{スポーツ管理 (CRUD)}
\begin{itemize}
    \item 大会で行うスポーツを登録・編集・削除できる。
    \item 各スポーツの概要、ルール、トーナメント表(後々アプリ側で完全ランダム生成するようにする)を登録できる。
\end{itemize}

\subsection{クラス・チーム管理}
参加単位となるクラスやチームを登録・管理する。

\subsection{学生管理}
\begin{itemize}
    \item ログインした全学生のリスト(表示名、メールアドレス)を閲覧できる。
    \item 学生の役割(一般学生 or 実行委員)を変更できる。
\end{itemize}

\subsection{参加者登録}
\begin{itemize}
    \item 各スポーツに、参加する学生をクラス単位または個人単位で割り当てる。
    \item UIは、クラスリストから参加するスポーツへドラッグ&ドロップするような直感的な操作を検討する。
\end{itemize}

\subsection{トーナメント・試合管理}
\begin{itemize}
    \item 登録された参加者情報を基に、トーナメント表を自動生成する。
    \item 試合結果(スコアなど)を入力し、トーナメント表に反映させる。勝者が自動的に次のコマに進むようにする。
    \item 試合結果に応じて、自動で得点付与処理を行う。
\end{itemize}

\subsection{QRコードリーダー機能}
\begin{itemize}
    \item デバイスのカメラを使用し、学生が提示するQRコードを読み取る。
    \item 読み取ったQRコードから学生情報を特定し、競技への参加受付(チェックイン)処理を行う。
\end{itemize}

\section{学生用アプリ機能要件}

\subsection{ダッシュボード}
自分が参加する競技の一覧や、本日の試合予定などを表示する。

\subsection{競技一覧・詳細閲覧}
\begin{itemize}
    \item 開催される全スポーツの一覧を閲覧できる。
    \item 各スポーツの詳細ページで、ルールや概要、現在のトーナメントの進捗状況を確認できる。試合結果もリアルタイムで反映される。
\end{itemize}

\subsection{QRコード発行機能}
\begin{itemize}
    \item ログイン中の学生に紐づいた、一意のQRコードを画面に表示する。
    \item このQRコードには、学生を識別するためのID情報(例: データベース上のユーザーID)が含まれる。
    \item 管理者がこのQRコードを読み取ることで、本人確認と参加登録が完了する。
\end{itemize}

\subsection{得点一覧閲覧}
\begin{itemize}
    \item 全クラスの現在の得点状況を閲覧できる。
    \item 自分のクラス名や競技名、現在の上位クラス等をソートして表示できる。
\end{itemize}

\section{データベース設計(案)}

\begin{table}[H]
    \centering
    \caption{主要なテーブル構成案}
    \label{tab:db_schema}
    \begin{tabular}{lp{6cm}p{6cm}}
        \toprule
        \textbf{テーブル名} & \textbf{カラム} & \textbf{説明} \\
        \midrule
        \texttt{users} & \texttt{id, google\_id, microsoft\_id, email, display\_name, role (admin or student), created\_at} & ユーザー情報。`role`で管理者と学生を区別する。 \\
        \addlinespace
        \texttt{sports} & \texttt{id, name, description, rules, format (tournamentなど)} & 競技情報。 \\
        \addlinespace
        \texttt{teams} & \texttt{id, name, class\_name} & 参加チーム(クラスなど)。 \\
        \addlinespace
        \texttt{participants} & \texttt{id, user\_id, team\_id} & ユーザーとチームを紐付ける中間テーブル。 \\
        \addlinespace
        \texttt{sport\_entries} & \texttt{id, sport\_id, team\_id} & どのチームがどの競技に参加するかを管理する。 \\
        \addlinespace
        \texttt{matches} & \texttt{id, sport\_id, round\_number, team1\_id, team2\_id, winner\_id, score, match\_time} & 試合情報。トーナメントの各試合を記録する。 \\
        \bottomrule
    \end{tabular}
\end{table}

\section{画面遷移図(案)}

\subsection{共通フロー}
\texttt{ログインページ} $\rightarrow$ \texttt{(Google/Microsoft認証)} $\rightarrow$ \texttt{認証成功} $\rightarrow$ \texttt{(初回ログインか?)}
\begin{itemize}
    \item \textbf{Yes}: \texttt{表示名設定ページ} $\rightarrow$ \texttt{(設定完了)} $\rightarrow$ \texttt{各アプリのダッシュボード}
    \item \textbf{No}: \texttt{各アプリのダッシュボード}
\end{itemize}

\subsection{管理者アプリ}
\texttt{ダッシュボード} $\rightarrow$ \texttt{スポーツ管理} / \texttt{参加者登録} / \texttt{試合結果入力} ...

\subsection{学生アプリ}
\texttt{ダッシュボード} $\rightarrow$ \texttt{得点閲覧} / \texttt{競技一覧} $\rightarrow$ \texttt{競技詳細・トーナメント表} / \texttt{QRコード表示}

\end{document}